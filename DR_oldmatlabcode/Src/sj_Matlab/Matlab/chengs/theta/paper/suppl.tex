%% $Id: suppl.tex,v 1.1 2007/05/15 20:06:03 chengs Exp $
%% notes.tex, Sen Cheng, 2004
%%
%% Notes for my own reference.


%\documentclass[11pt]{article}
%\usepackage{pslatex}

\documentclass[preprint,rmp,preprintnumbers,floatfix]{revtex4}
%\documentclass[twocolumn,rmp,preprintnumbers,floatfix]{revtex4}
\usepackage{graphicx,amsmath,amsfonts}


\usepackage{rotating}
% keyboard macros
\newcommand{\pval}[1]{$p= #1$}
\newcommand{\pvalgr}[1]{$p > #1$}
\newcommand{\pvallt}[1]{$p < #1$}
\newcommand{\ranksum}{Symbols mark results of rank-sum test:  * \pvallt{0.05},
** \pvallt{0.01}, *** \pvallt{0.001}.\ }
\renewcommand{\eqref}[1]{Eq.~\ref{#1}}
\newcommand{\eqsref}[2]{Eqs.~\ref{#1}~and~\ref{#2}}
\newcommand{\figref}[1]{Fig.~\ref{#1}}
\newcommand{\de}[1]{\,{\rm d}#1}
\newcommand{\DD}[1]{\frac{d}{d #1}}
\newcommand{\dd}[1]{\frac{\partial}{\partial #1}}
\newcommand{\ddz}[2]{\frac{\partial #1}{\partial #2}}

\setlength{\parindent}{0mm}

%\setlength{\voffset}{-1.5in} 
%\setlength{\hoffset}{-1in}
%\setlength{\evensidemargin}{15mm} 
%\setlength{\oddsidemargin}{15mm} 
%\setlength{\topmargin}{15mm} 

%\setlength{\textwidth}{183mm} 
%\setlength{\textheight}{247mm} 

\graphicspath{{/home/chengs/figs/papers/07-MemoryDynamics-Nature/}{/home/chengs/figs/papers/}}

\bibliographystyle{apalike} % has to come before \begin{document} in revtex4

\renewcommand{\refname}{\Large Reference List}
\renewcommand{\figurename}{{Figure}}
%\pagestyle{empty} %% @@

\begin{document}



% all figure produced with   pf9, xp6,t005 , 



%\title{Supplementary Information}
%\maketitle

\noindent{\bf \Large Supplementary Information}

\section*{\large Supplementary Methods}

{\it Details of Adaptive Filtering Algorithm.}\\

Standard histogram approaches are not adequate for quantifying moment-by-moment
changes in neural activity, particularly when the animal's behaviour is not
uniform in space and time.  We therefore derived an adaptive filtering
algorithm \cite{Brown2001,Frank2002} to estimate the changing
relationship between position $x_t$, theta phase $\theta_t$ and firing rate
$\lambda(t)$.  
%
We assumed a flexible model for the instantaneous spike rate $\lambda(t)$ that
evolves as the neuron's spiking changes. Two factors contribute, the
position-phase response $\lambda_S$ and the interspike interval (ISI)
distribution $\lambda_T$ such that:
$\lambda(t;\mathbf{q}_t)= 
\lambda_S(x_t,\theta_t;\mathbf{q}_t) \lambda_T(\tau;\mathbf{q}_t)$
%
The functions $\lambda_S$ and $\lambda_T$ are parameterized by
cardinal cubic splines, whose parameters are combined in the parameter vector
$\mathbf{q}_t$. These splines make it possible to capture the fine spatial and
temporal structure of phase precession and the temporal structure of refractory
period and bursting.  
%
The parameters of the splines functions correspond to the
functional values at the control points. For the position-phase response
function the control points were spaced equidistantly in position (every
$\sim$5cm) and theta phase (every 30$^\circ$). The control points of the ISI
distribution were spaced unevenly (0ms, 1ms, 3ms, \dots , 11ms, 15ms, 20ms,
\dots , 65ms, 80ms) to give greater emphasis to the refractory period and
bursting. For ISI's larger than 80ms, the ISI distribution was set to 1 to avoid
confounding effects of theta modulation ($\sim$~125ms). An adaptive filter was
derived for estimating the parameters
$\mathbf{q}_t$ in timesteps of $\Delta t=2$~ms:
\begin{equation}
    \mathbf{q}_{t+1}= \mathbf{q}_{t} + \mathbf{\varepsilon} \cdot 
    \dd{\mathbf{q}} \lambda(t;\mathbf{q}_t) \left[ dN_t -
    \lambda(t;\mathbf{q}_t) \Delta t \right]
\end{equation}
The learning rates $\varepsilon$ were 6 and 0.05 for the parameters of the
spatial and temporal functions, respectively.  While the learning rates
determine how quickly estimates of the cell's response properties change, we
verified that our conclusions about phase precession did not differ
qualitatively even if much larger or smaller learning rates were used.
\\

The quality of the fit was assessed by a Kolmogorov-Smirnov test (KS) on
time-rescaled interspike times \cite{Brown2002,Frank2002}.
Including phase precession substantially improved the fit over the previous
model that did not take phase modulation into account. For 131 of 185 cells
(71\%) from the first session (training configuration) the KS statistics was
within the 99\%-confidence interval. In the second session (novel configuration),
104 of 191 cells (55\%) were within that bound. These results indicate that the
model accurately captures much of the spatio-temporal structure of place cell
activity.  In addition, these estimated models could also be used to generate
data in order to accurately simulate CA1 place cell responses.  To our
knowledge, our current model is the most accurate to date for data analysis or
simulation of place cell activity.

\begin{widetext}
\end{widetext}
\mbox{}
%\begin{figure*}[htpb]
%\end{figure*}

\clearpage
\section*{\large Movie Caption}

\noindent {\bf Dynamic estimate of position-phase response of one neuron with
place field in novel arm.}  (Length 57s)

{\it Bottom right}: Top-down view of the maze in a novel
configuration. The left outer arm is novel; the right outer arm is familiar. A
white square tracks the location of the animal.  

{\it Bottom left}: Interspike interval (ISI) distribution estimated by the
adaptive filter (see Supplementary Methods). Time since last spike in seconds is
on the x-axis. The y-axis shows the temporal component of the firing rate
function.  The firing rate is the product of the current interspike interval
intensity and the current location--phase intensity.

{\it Top}: The four panels show the position-phase intensity for four movement
trajectories defined as follows (from top to bottom): 0: home to left outer arm,
1: left outer arm to home, 2: home to right outer arm, and 3: right outer arm to
home. Within each panel theta phase in radians is on the x-axis and position in
cm is on the y-axis.  The left halves of all four panels correspond to same
locations in the home arm. The right halves of the two top panels correspond to
same locations in the novel arm, and the right halves of the two bottom panels
correspond to same locations in the familiar arm.  A moving bar tracks the
current position of the animal. When the cell fires a spike, the position and
theta phase of the spike are indicated by small white dots. Time is indicated in
the top right hand corner. 
\\

This movie shows the evolution of phase precession for one neuron that developed
a place field on the novel arm.  Phase precession on trajectory 0 and 1 (top two
panels) was weak initially but then became stronger over time. This apparent
trend is captured by our measure for phase precession strength (Fig. 3c shows
this measure for the place field on trajectory 0). Note that the place field in
the novel arm is bidirectional, i.e., the place fields are visible in the right
halves of the two top panels. The place field exhibits dynamic theta phase
precession in both panels with opposite slopes, appropriate for the running
direction in each case.  The interspike interval structure is much less dynamic,
suggesting that the short interspike interval structure of CA1 spike trains is
relatively stable despite the large changes in phase precession.  

\begin{widetext}
\end{widetext}
\mbox{}

\clearpage
\section*{\large Supplementary Figures}

\setcounter{figure}{0}
\renewcommand\thefigure{{S\arabic{figure}}}

\begin{figure}[htbp]
    \centering
        \includegraphics{bar_ripRate_CA1PE}
    \caption{ {\bf Ripples occur more frequently in novel arm than in familiar
    arm.} 
    Average number of ripples per time that the animal spent in the novel
    arm (red, green, and blue bars) and in the familiar arm (black bars) across
    the three days of exposure to the novel arm. Error bars represent standard
    errors across sessions. \ranksum Only within-day planned comparisons were performed.
    }
\end{figure}

\newpage %%@@
\begin{figure}[htbp]
    \centering
    \parbox[t]{3in} {
        \rule{.9in}{1pt} \hfill
        novel arm 
        \hfill \rule{0.9in}{1pt} \\
        \parbox[t]{1.4in} { {\sf a}\hfill \mbox{} \\
            \includegraphics{bar_ripCoincFrac_overlap_novelArm_pf9}
        } \hfill
        \parbox[t]{1.4in} { {\sf b} \hfill \mbox{} \\
            \includegraphics{bar_z_overlap_novelArm_pf9}
        }
    } \\ \vskip5mm
    \parbox[t]{3in} {
        \rule{.8in}{1pt} \hfill
        familiar arm 
        \hfill \rule{0.8in}{1pt} \\
        \parbox[t]{1.4in} { {\sf c} \hfill \mbox{} \\
            \includegraphics{bar_ripCoincFrac_overlap_famArm_pf9}
        } \hfill
        \parbox[t]{1.4in} { {\sf d} \hfill \mbox{} \\
            \includegraphics{bar_z_overlap_famArm_pf9}
        }
    }
    \caption{  {\bf Co-activity of novel arm cell pairs during ripple is specific.}
    If the greater co-activity of novel arm cells as compared to familiar arm
    cells carries a meaningful signal, then pairs consisting of one novel and
    one familiar arm cell should not show strong co-activity.  
    {\bf a,c,} The probability of a pair being co-active during any given ripple
    and, 
    {\bf b,d,} the z-score for coordinated co-activity (see Methods) in the
    novel arm and in the familiar arm.  
    Red, green and blue bars represent pairs of novel arm cells on days 1, 2 and
    3, respectively.  Gray bars represent results for pairs made up of one novel
    and one familiar arm cell.  
    Error bars represent standard errors.  \ranksum
    Only within-day
    planned comparisons were performed. 
    Mixed pairs were significantly less co-active and coordinated during ripples
    than were pairs of novel arm cells.
    }
\end{figure}

%\newpage %%@@
\begin{figure*}[htbp]
    \centering
        \parbox[t]{3in} { {\sf a}  \hfill \mbox{} \\
            \includegraphics{cdf-p_excessCorr_pf9_overlap_nonripple_famArm123}
        }
        \parbox[t]{3in} { {\sf b}  \hfill \mbox{} \\
            \includegraphics{cdf-p_excessCorr_pf9_overlap_placefield_famArm123}
        }
    \caption{{\bf  Higher excess correlation in novel arm cell pairs is solely due to ripple activity.   }
    To determine the factors that lead to higher excess correlation in novel arm
    cells, we examined the effect of excluding spikes within ripple or using
    within-place-field spikes alone.  We found that spiking during ripples was
    responsible for the excess correlation, in that differences between novel
    and familiar arm cell pairs disappeared 
    {\bf a,} when spikes fired during ripples
    were excluded, day 1 (rank-sum test, \pval{0.58}), day 2 (\pval{0.30}), or
    day 3 (\pval{0.58}) and 
    {\bf b,}  when only spikes fired within place fields were included,
    day 1 (\pval{0.87}), day 2 (\pval{0.63}), or day 3 (\pval{0.38}).
    }
\end{figure*}

\newpage %%@@
\begin{figure*}[htbp]
    \centering
%    \includegraphics[width=4.5in]{cdf_excessCorr_pf9_overlap_ripple_std3_6}
    \includegraphics{cdf_excessCorr_pf9_overlap_ripple_std3_6}
    \caption{{\bf  Effect of ripple detection threshold on excess correlation. }
    We compared the excess correlation for novel arm cell pairs on day 1 (red
    lines) and familiar arm cell pairs across all days (black lines) in three
    conditions to understand how the threshold for ripple detection affects our
    conclusions.  In the first condition, all of the spikes were included (solid
    red and black lines).  In the second and third conditions, only spikes that
    occur outside of ripples were included, but different thresholds for ripple
    detection were used.  In the second condition we used a threshold of mean +
    3 standard deviations (SD) for the amplitude envelop (see Methods) while in
    the third condition we used a threshold of mean + 6 SD.  Note that the
    higher the threshold, the fewer ripples are detected, and as a result fewer
    spikes were included. When the higher threshold was used for ripple
    detection (dash-dotted lines), the excess correlation of non-ripple spiking
    was higher in novel than in familiar arm cell pairs (rank-sum test,
    \pval{0.016}). However, with the lower threshold (dotted lines) more potential
    ripples were detected and there was no longer a difference in excess
    correlation when ripples were excluded (\pval{0.58}). 
    %
    Excluding ripples of any amplitude did not affect the
    excess correlation in familiar arm cell pairs.  These results suggest that
    while a high ripple threshold will ensure the rejection of non-ripple
    events, it also leads to the rejection of many low-amplitude ripple events
    associated with strong correlations between neurons. To our knowledge, these
    results represent the first localization of a reactivation effect within
    ripples, as several previous studies used a more exclusive criterion and
    reported that reactivation occurred outside ripples
    \cite{Wilson1994,Lee2002,Jackson2006,Foster2006}.  
    }
\end{figure*}

\newpage %%@@
\begin{figure*}[htbp]
    \centering
    \includegraphics{cdf-p_mean_pf9_overlap_ripple_famArm123}
    \caption{{\bf  Coordinated spiking during ripples is temporally symmetric.}
    To test whether ripple activity reflected forward \cite{Skaggs1996b,Lee2002}
    or reverse replay \cite{Foster2006}, we calculated the mean time lag after
    normalizing the cross-correlations
    between $\pm$100ms. All spike pairs where one or both of the spikes was fired
    during a ripple were included. Only cell pairs with 10 or more coincident
    events were included in analysis. The mean time lag between two neurons
    should be correlated with the order in which their place fields were
    traversed if there was consistent forward or reverse replay and the place
    fields were unidirectional.  In this experiment place fields were generally
    bidirectional, so place fields were traversed in both orders. This makes it
    impractical to order neurons by the location of their place fields. We
    therefore ordered neurons according to the location of the largest peak in
    the cross-correlogram including all spikes (CCG-all). If place cells were
    unidirectional, the spatial order of their place fields would correspond to
    the sign of the CCG-all peak. The distributions of mean time lag do not
    significantly deviate from normal (Lilliefors test, \pvalgr{0.2}), except for novel
    arm cell pairs on day 1, which deviates from normal due to only two outliers
    (\pvallt{0.01}, \pvalgr{0.20} if two outliers are excluded). We therefore use the t-test
    for zero mean and find that none of the distributions were significantly
    different from zero:  familiar arm (\pval{0.15}), novel arm cell pairs on day 1
    (\pval{0.16}), day 2 (\pval{0.13}), day 3 (\pval{0.11}).  Thus, there is no evidence for a
    preference for either forward or reverse replay in our data.  
    }
\end{figure*}

\newpage %%@@
\begin{figure*}[htbp]
    \centering
    \parbox[t]{1.5in}{
        \parbox[t]{1.5in}{ {\sf a}  \hfill \mbox{} \\
            \includegraphics{EEG-snippet1}
        }
        \parbox[t]{1.5in}{ {\sf b}  \hfill \mbox{} \\
            \includegraphics{pmtm-kyl-01-4-28-novelArm}
        }
    }
    \parbox[t]{4.6in}{
        \parbox[t]{1.5in} { {\sf c} \hfill day 1\hfill\mbox{} \\
            \includegraphics{cdf-p_EEG-theta_all_day1}
        }
        \parbox[t]{1.5in} { \hfill day 2\hfill\mbox{} \\
            \includegraphics{cdf-p_EEG-theta_all_day2}
        }
        \parbox[t]{1.5in} { \hfill day 3\hfill\mbox{} \\
            \includegraphics{cdf-p_EEG-theta_all_day3}
        }
        \parbox[t]{1.5in} { {\sf d}  \hfill \mbox{} \\
            \includegraphics{cdf-p_EEG-width_all_day1}
        }
        \parbox[t]{1.5in} { \mbox{} \\
            \includegraphics{cdf-p_EEG-width_all_day2}
        }
        \parbox[t]{1.5in} { \mbox{} \\
            \includegraphics{cdf-p_EEG-width_all_day3}
        }
        \parbox[t]{1.5in} { {\sf e}  \hfill \mbox{} \\
            \includegraphics{cdf-p_EEG-max_all_day1}
        }
        \parbox[t]{1.5in} { \mbox{} \\
            \includegraphics{cdf-p_EEG-max_all_day2}
        }
        \parbox[t]{1.5in} { \mbox{} \\
            \includegraphics{cdf-p_EEG-max_all_day3}
        }
    }
    \caption{{\bf Theta oscillations are similar across novel and familiar arms.   }
    {\bf a,} Sample of unfiltered local field potential (LFP) recording obtained
    while the animal was located in the novel arm. LFP recordings were
    normalized by the root mean squared amplitude to normalize for different
    gains across
    tetrodes. Periodic oscillations at about 8Hz are clearly visible. 
    {\bf b,} Representative power spectral density (PSD) of normalized local
    field potential for one tetrode in one session, including only times when
    animal was located in novel arm. Note the clear peak around 8Hz. PSD were
    calculated on normalized LFP using multi-taper method with minimum time
    window width of 1s. Only times when animals were running (speed $>$ 4pixels/s=
    2.8cm/s) were included.  We then determined the PSD for every tetrode that
    contributed at least one neuron with an identified place field. The PSD peak
    within the theta range 6--10Hz was then analyzed with regard to frequency,
    half-width-at-half-height, and height.  We compared times when animal was in
    a novel arm (red, green, and blue) to times in the familiar arm (black)
    across days 1--3. p-values are for rank-sum tests. 
    {\bf c,} The cumulative distribution shows that the peak frequency of theta
    is tightly distributed between 7--8Hz, and did not differ between novel and
    familiar arms on any of the three days. 
    {\bf d,} The theta peak was as sharp in the novel as in the
    familiar arm on all three days.  
    {\bf e,} Theta power on day 1 was lower in novel arm, but not on days 2 and
    3.  The difference on day 1 lead us to analyse
    the relationship between theta power and phase precession strength for day 1
    (see Figure \ref{fig:corr_V_peak}).
    }
\end{figure*}

\clearpage %%@@
\begin{figure}[htbp]
    \centering
    \includegraphics{corr_V_peak_day1}
    \caption{{\bf  Lower theta power on day 1 does not account for differences in phase precession.  }
    The strength of theta phase precession was lower in the novel arm than in
    the familiar arm only during the initial exposure on day 1. To examine
    whether this difference might be due in part to lower theta power in the
    novel arm, we analyzed the relationship between phase precession strength
    and theta power during the first minute of exposure. Clearly, phase
    precession was weaker and theta power was lower on the novel arm (red
    points) as compared to the familiar arm (black points). However, there was
    no significant linear relationship between phase precession strength and
    theta power in the familiar arm (F-test, \pval{0.49}), and only a weak one in
    the novel arm (\pval{0.024}), accounting for only 7.9\% of the variance. These
    results indicate that theta power is only weakly, if at all, related to the
    strength of phase precession. 
    \label{fig:corr_V_peak} }
\end{figure}

\newpage %%@@
\begin{figure*}[htbp]
    \centering
    \parbox[t]{4.6in} {
        \parbox[t]{1.5in} { {\sf a}  \hfill \mbox{} \\
            \includegraphics{novel-1-27-kyl-8-4-25-1}
            }
        \parbox[t]{1.5in} { {\sf b}  \hfill \mbox{} \\
            \includegraphics{fam-1-18-kyl-4-4-27-2}
        }
        \hfill \mbox{} \\   \vskip5mm
        \parbox[t]{1.5in} { {\sf c} \hfill day 1\hfill\mbox{} \\
            \includegraphics{cdf-p_thetaPower_pf9_none_placefield_day1}
        }
        \parbox[t]{1.5in} { \hfill day 2\hfill\mbox{} \\
            \includegraphics{cdf-p_thetaPower_pf9_none_placefield_day2}
        }
        \parbox[t]{1.5in} { \hfill day 3\hfill\mbox{} \\
            \includegraphics{cdf-p_thetaPower_pf9_none_placefield_day3}
        }
    }
    \caption{{\bf  Theta modulation of autocorrelations is not different for novel and familiar arm cells.  }
    Auto-correlogram (ACG) of a neuron with place field in 
    {\bf a,} novel arm on day 1
    and 
    {\bf b,} familiar arm. Only spikes fired within place fields were included.
    ACG was calculated in 2ms time bins.  We then calculated the power
    spectral density of the ACG using the multi-taper method and integrated the
    power in the theta band (6--10 Hz). 
    {\bf c,} Shown is theta power for novel (red,
    green, blue) and familiar (black) arm cells on days 1--3. p-values are shown
    for rank-sum test.  Novel arm cells were at least as strongly theta
    modulated as familiar arm cells, indicating that the observed weaker phase
    precession on day 1 was not due to lower overall theta modulation for novel
    arm cells.
    }
\end{figure*}


\newpage %%@@
\begin{figure*}[htbp]
    \centering
    \parbox[t]{6.1in} { 
        \parbox[t]{2in} { \mbox{} \hfill day 1\hfill\mbox{} \\
            \includegraphics{bar_vel_moving2_minocc_all_day1}
        }
        \parbox[t]{2in} { \mbox{} \hfill day 2\hfill\mbox{} \\
            \includegraphics{bar_vel_moving2_minocc_all_day2}
        }
        \parbox[t]{2in} { \mbox{} \hfill day 3\hfill\mbox{} \\
            \includegraphics{bar_vel_moving2_minocc_all_day3}
        }
    }
    \caption{{\bf Running speed differs across all three days of exposure.   }
    Given that the animals' behavior changes as a result of learning, we asked
    whether differences in behavior across days could account for the observed
    differences in phase precession or sequence compression. As a behavioral
    variable we choose running speed which is shown (mean$\pm$SE) for the first
    three minutes of experience in novel (red, green, and blue bars) and
    familiar (black bars) arms. The animals ran significantly more slowly during
    the first minute of experience in the novel arm as compared to the familiar
    arm on all days (rank-sum test, \pvallt{0.007}).  This result shows that animals
    clearly distinguished between the novel and the familiar arm. In contrast,
    phase precession differences were no longer present on day 2 and sequence
    compression differences were no longer present by day 3.  The finding that
    phase precession did not differ between novel and familiar arms on days 2
    and 3 even though the animals' speeds were significantly different, confirms
    previous observations that running speed does not strongly modulate phase
    precession \cite{Huxter2003}.  Finally, we note that even if some
    behavioral (or, for that matter, physiological) variables accounted for all
    the differences in phase precession or sequence compression, the fact would
    remain that spatial activity {\it is} less coordinated in novel
    environments.  
    In other words, the conclusion that coordinated ripple reactivation is
    not a simple replay of spatial activity stands regardless of the cause of
    the less coordinated spatial activity.  
    }
\end{figure*}



\newpage %%@@
\begin{figure*}[htbp]
    \centering
%\flushleft
    \parbox[t]{3in} { {\sf a} \hfill\mbox{} \\
        \includegraphics{coefAbs_pf9_fam1}
    }
    \parbox[t]{3in} { 
        \mbox{}
    } \\
    \parbox[t]{3in} { {\sf b} \hfill\mbox{} \\
        \includegraphics{seq_comp_pf9_fam_redux}
    }
    \parbox[t]{3in} { {\sf c}\hfill\mbox{}  \\
        \includegraphics{cdf-p_excessCorr_pf9_overlap_all_famfamArm123}
    }
    \caption{{\bf  Neural representation of the familiar arm remains stable.}
    We assessed the stability of the hippocampal representation by comparing the
    neural activity of familiar arm cells to that of cells with place fields in
    the outer arms of the training configuration, which animals experience in
    the first session every day (see Methods). 
    {\bf a,} Phase precession in training configuration was initially stronger
    than in the novel arm on day 1 (n=280, 64, \pval{0.0014} at t=1s) but not
    different from that in the familiar arm (rank-sum test, n=280, 49,
    \pval{0.28} at t=1s). 
    {\bf b,} Sequence compression in familiar configuration.  The sequence
    compression index (SCI) was not different from SCI in familiar arm cell
    pairs (z-test, \pval{0.48}). It is different from SCI in novel arm cell pairs on
    day 1 (\pval{0.003}) and day 2 (\pval{0.002}), but not on day 3
    (\pval{0.90}) ---
    consistent with the direct comparison between novel and familiar arm cell
    pairs. 
    {\bf c,} Excess correlation in familiar arm cell pairs was stable across
    days 1--3 and not significantly different from excess correlation in the
    training configuration on day 1 (rank-sum test, \pval{0.17}) and day 2
    (\pval{0.78}).
    Although the day 3 comparison (\pval{0.041}) is significant at the 0.05-level,
    the difference in excess correlation is very small.
    }
\end{figure*}



\clearpage  %@@

%%% Create the reference section using BibTeX:
\bibliography{stat,hippo}


\end{document}


\begin{figure}[htbp]
    \centering
    \includegraphics[width=4in]{figs/}
    \caption{
    \label{fig:}
    }
\end{figure}


